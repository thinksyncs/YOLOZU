\chapter{Score Calibration and Long-tail Adjustments}

\ChapterMeta{
This chapter describes post-hoc calibration and long-tail adjustments that transform confidence scores while preserving geometry.
}{
It can improve thresholding behavior and long-tail metrics without retraining, while keeping the transform reproducible via stats artifacts.
}{
It requires a dataset and a predictions artifact; method-specific parameters apply, and outputs should typically be written as separate calibrated artifacts.
}

\section{When calibration helps}
If your model is accurate but systematically miscalibrated (scores are too high/low), or if
long-tail classes are under-confident, post-hoc calibration can improve thresholding behavior and
long-tail metrics \textit{without retraining}.

\section{The calibrate command}
\yolozu{} supports multiple methods under a unified interface.
Conceptually, calibration modifies \textbf{confidence scores} while leaving geometry (bbox/mask/keypoints) unchanged.
The CLI writes both the calibrated predictions artifact and a small stats/report JSON so the transform is reproducible.

Example (FRACAL, bbox):
\begin{lstlisting}[language=bash]
yolozu calibrate \
  --method fracal \
  --task bbox \
  --dataset /path/to/yolo-dataset \
  --predictions reports/predictions.json \
  --out reports/predictions_calibrated.json \
  --stats-out reports/fracal_stats_bbox.json
\end{lstlisting}

Reuse stats later:
\begin{lstlisting}[language=bash]
yolozu calibrate \
  --method fracal \
  --task bbox \
  --dataset /path/to/yolo-dataset \
  --predictions reports/predictions.json \
  --out reports/predictions_calibrated.json \
  --stats-in reports/fracal_stats_bbox.json
\end{lstlisting}

\section{Supported methods (high level)}
\begin{itemize}
  \item \textbf{FRACAL}: frequency-aware calibration using class count statistics (repo method; compare alongside standard calibration baselines).
  \item \textbf{LA (logit adjustment)}: apply a class-prior logit shift (parameter \cmd{--tau}) \cite{menon2021longtail}.
  \item \textbf{NorCal}: frequency-based adjustment (parameter \cmd{--gamma}).
  \item \textbf{Temperature scaling}: global temperature \cmd{--temperature}, with optional fitting for certain tasks \cite{guo2017calibration}.
\end{itemize}

\section{Tasks}
Calibration is task-aware:
\begin{itemize}
  \item \textbf{bbox}: adjusts detection scores.
  \item \textbf{seg}: adjusts instance segmentation scores (masks are unchanged).
  \item \textbf{pose}: adjusts detection scores; keypoints are preserved (no geometric change).
\end{itemize}

\section{Operational tips}
\begin{itemize}
  \item Prefer keeping calibrated outputs as separate artifacts.
  \item Record the method and parameters (the CLI writes a report).
  \item Compare methods side-by-side on the same predictions.
\end{itemize}
