\chapter{Score Calibration and Long-tail Adjustments}

\section{When calibration helps}
If your model is accurate but systematically miscalibrated (scores are too high/low), or if
long-tail classes are under-confident, post-hoc calibration can improve thresholding behavior and
long-tail metrics \textit{without retraining}.

\section{The calibrate command}
\yolozu{} supports multiple methods under a unified interface.
See \path{docs/score_calibration.md} for the most up-to-date details.

Example (FRACAL, bbox):
\begin{lstlisting}[language=bash]
yolozu calibrate \
  --method fracal \
  --task bbox \
  --dataset /path/to/yolo-dataset \
  --predictions reports/predictions.json \
  --out reports/predictions_calibrated.json \
  --stats-out reports/fracal_stats_bbox.json
\end{lstlisting}

Reuse stats later:
\begin{lstlisting}[language=bash]
yolozu calibrate \
  --method fracal \
  --task bbox \
  --dataset /path/to/yolo-dataset \
  --predictions reports/predictions.json \
  --out reports/predictions_calibrated.json \
  --stats-in reports/fracal_stats_bbox.json
\end{lstlisting}

\section{Supported methods (high level)}
\begin{itemize}
  \item \textbf{FRACAL}: frequency-aware calibration using class count statistics.
  \item \textbf{LA (logit adjustment)}: apply a class-prior logit shift (parameter \cmd{--tau}).
  \item \textbf{NorCal}: frequency-based adjustment (parameter \cmd{--gamma}).
  \item \textbf{Temperature scaling}: global temperature \cmd{--temperature}, with optional fitting for certain tasks.
\end{itemize}

\section{Tasks}
Calibration is task-aware:
\begin{itemize}
  \item \textbf{bbox}: adjusts detection scores.
  \item \textbf{seg}: adjusts instance segmentation scores (masks are unchanged).
  \item \textbf{pose}: adjusts detection scores; keypoints are preserved (no geometric change).
\end{itemize}

\section{Operational tips}
\begin{itemize}
  \item Prefer keeping calibrated outputs as separate artifacts.
  \item Record the method and parameters (the CLI writes a report).
  \item Compare methods side-by-side on the same predictions.
\end{itemize}
