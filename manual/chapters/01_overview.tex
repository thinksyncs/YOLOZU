\chapter{Overview}

\ChapterMeta{
This chapter introduces \yolozu{}'s contract-first evaluation paradigm, delineates the primary repository entry points, and clarifies the distinction between its two Command Line Interface (CLI) surfaces.
}{

It is designed to expedite workflow selection and eliminate ambiguity regarding command execution and artifact management.
}{
No prerequisites are required beyond reading. Rely on the \cmd{--help} flag for the specific CLI in use, and treat JSON artifacts as the definitive, stable interface.
}

\section{What \yolozu{} Is}
\yolozu{} operates as a \textbf{contract-first evaluation harness}.
The foundational philosophy is straightforward: \textit{execute inference in any environment}, export a standardized \path{predictions.json} artifact, and subsequently perform consistent, reproducible evaluations.


\section{AI-First Design Intent}
This repository is architected as an \textbf{AI-first} codebase.
In practical terms, this implies:
\begin{itemize}
  \item AI agents must be capable of \textbf{utilizing} the codebase safely via stable command surfaces and well-defined contracts.
  \item AI agents must be able to \textbf{extend} workflows (e.g., introducing new experiments, scripts, or reports) without violating existing interfaces.
  \item JSON schemas and machine-readable registries are treated as first-class interfaces, rather than optional metadata.
\end{itemize}

Operational guidance for both human operators and agentic workflows dictates that artifacts must be treated as first-class entities, contracts must be validated early in the pipeline, and evaluation protocols must remain strictly pinned.

High-level objectives include:
\begin{itemize}
  \item Generating stable, versioned JSON artifacts (encompassing predictions, evaluation reports, and run metadata).
  \item Ensuring CPU-minimum compatibility for development and testing; GPU acceleration remains optional for training and inference.
  \item Facilitating backend-agnostic workflows (supporting PyTorch, ONNX Runtime, TensorRT, and custom C++/Rust implementations).
  \item Streamlining migration by accepting prevalent dataset ecosystems (e.g., Ultralytics/YOLO formats, COCO JSON) via lightweight wrapper artifacts, thereby preserving the integrity of original datasets.
  \item Enforcing a production-oriented run contract for training operations, which standardizes artifact paths, resumption capabilities, and model exports.
  \item Maintaining a strict Apache-2.0 operational posture, ensuring no non-Apache dependencies are present in shipped code paths.
\end{itemize}


\section{Navigating the Repository}
The primary entry points for navigating the project are:
\begin{itemize}
  \item \path{README.md}: Provides a concise overview and essential canonical commands.
  \item \path{docs/README.md}: Serves as the documentation index, organized by \textit{task objective}.
  \item \path{docs/yolozu_spec.md}: Contains the comprehensive feature summary and technical specification.
  \item \path{tools/}: Houses power-user scripts and the repository's wrapper CLI.
\end{itemize}


\section{Dual CLI Surfaces: Pip vs. Repository}
\yolozu{} exposes two distinct command surfaces to accommodate different user needs:
\begin{itemize}
  \item \textbf{Pip CLI}: Invoked via \cmd{yolozu ...}, this is the install-safe, user-facing command interface.
  \item \textbf{Repository CLI}: Invoked via \cmd{python3 tools/yolozu.py ...}, this interface is tailored for research, evaluation, and advanced workflows.
\end{itemize}

Both CLIs share underlying concepts and data contracts. This manual references whichever CLI is most appropriate for the specific workflow under discussion.
