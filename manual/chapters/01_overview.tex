\chapter{Overview}

\section{What \yolozu{} is}
\yolozu{} is a \textbf{contract-first evaluation harness}.
The core idea is: \textit{run inference anywhere}, export a stable \path{predictions.json},
then evaluate consistently.

\section{AI-first design intent}
This repository is organized as an \textbf{AI-first} codebase.
In practice, that means:
\begin{itemize}
  \item AI agents should be able to \textbf{use} the code safely via stable command surfaces and contracts.
  \item AI agents should be able to \textbf{extend} workflows (new experiments, new scripts, new reports) without breaking interfaces.
  \item JSON schemas and machine-readable registries are treated as first-class interfaces, not optional metadata.
\end{itemize}

See \path{docs/ai_first.md} for the operational guidance used by agentic workflows.

Key goals (high-level):
\begin{itemize}
  \item Stable, versioned JSON artifacts (predictions, reports, run metadata).
  \item CPU-minimum development and tests; GPU is optional for training/inference.
  \item Backend-agnostic workflows (PyTorch / ONNX Runtime / TensorRT / custom C++/Rust).
  \item Easy migration: accept common dataset ecosystems (Ultralytics/YOLO, COCO JSON) via small wrapper artifacts, without modifying your original dataset.
  \item Production-oriented run contract for training (artifact paths, resume, export).
  \item Apache-2.0-only operational posture (see \path{docs/license_policy.md}).
\end{itemize}

\section{How to navigate the repository}
The main entry points are:
\begin{itemize}
  \item \path{README.md}: quick overview and a few canonical commands.
  \item \path{docs/README.md}: docs index organized by \textit{what you want to do}.
  \item \path{docs/yolozu_spec.md}: feature summary/spec.
  \item \path{tools/}: power-user scripts and the repository wrapper CLI.
\end{itemize}

\section{Two CLIs: pip vs repo}
\yolozu{} typically exposes two command surfaces:
\begin{itemize}
  \item \textbf{pip CLI}: \cmd{yolozu ...} (install-safe, user-facing commands)
  \item \textbf{repo CLI}: \cmd{python3 tools/yolozu.py ...} (research/eval workflows)
\end{itemize}

They share concepts and contracts. This manual uses whichever is most common for the workflow
being discussed.
