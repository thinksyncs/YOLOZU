\chapter{CLI Reference (Practical)}

\ChapterMeta{
This chapter provides a concise summary of the most frequently utilized commands and demonstrates how to explore the complete Command Line Interface (CLI) surface via the \cmd{--help} flag.
}{
It is designed to minimize the time to first execution and to clearly delineate which commands are install-safe versus those restricted to repository-only workflows.
}{
Utilize \cmd{yolozu --help} for the pip-installed CLI, or \cmd{python3 tools/yolozu.py --help} for the repository wrapper. Because implementation details evolve, the \cmd{--help} output remains the definitive source of truth.
}

\section{Design Philosophy}
The CLI surface is intentionally bifurcated to serve distinct user profiles:
\begin{itemize}
  \item \cmd{yolozu}: The end-user safe CLI, installed globally or virtually via pip.
  \item \cmd{python3 tools/yolozu.py}: The repository wrapper, specifically tailored for research and evaluation pipelines.
\end{itemize}

This chapter documents the \textit{high-traffic commands} and outlines the methodology for discovering the broader command set.

\section{Discoverability}
As a best practice, always initiate discovery with:\\
\cmd{yolozu --help} and \cmd{yolozu <command> --help}.\\
For repository-specific workflows, utilize: \cmd{python3 tools/yolozu.py --help}.

\section{Common Commands}
\begin{longtable}{@{}p{0.40\textwidth}p{0.55\textwidth}@{=}
\toprule
\textbf{Command} & \textbf{Purpose} \\
\midrule
\cmd{yolozu doctor} & Executes environment diagnostics (verifying dependencies, GPU status, and backend availability). \\
\cmd{bash scripts/smoke.sh} & Runs the one-command offline smoke flow (doctor + validate + eval-coco dry-run on \path{data/smoke}). \\
\cmd{yolozu train} & Initiates YAML/JSON config-driven RT-DETR training (fully supporting the run contract and resumption capabilities). \\
\cmd{yolozu test} & Executes the YAML/JSON scenario runner (compatible with dummy, precomputed, and rtdetr\_pose adapters). \\
\cmd{yolozu predict-images} & Performs folder-level inference, generating predictions, visual overlays, and an HTML summary. \\
\cmd{yolozu eval-coco} & Conducts COCO mAP evaluation derived from \path{predictions.json} (requires COCO tooling to be installed). \\
\cmd{yolozu parity} & Compares two distinct predictions artifacts to detect backend drift or discrepancies. \\
\cmd{yolozu calibrate} & Applies score calibration and long-tail post-hoc adjustments. \\
\cmd{python3 tools/eval_suite.py} & Executes suite evaluation, providing protocol-pinned reporting and cross-run comparisons. \\
\cmd{python3 tools/hpo_sweep.py} & Orchestrates hyperparameter sweeps and aggregates metrics into CSV, MD, or JSONL formats. \\
\cmd{python3 tools/benchmark_latency.py} & Runs the latency/FPS benchmark harness, maintaining a JSONL history to track performance regressions. \\
\cmd{python3 tools/validate_tool_manifest.py} & Validates the structural integrity of the machine-readable registry (\path{tools/manifest.json}). \\
\bottomrule
\end{longtable}

\section{YAML Config-Driven Usage (Train/Test/Resume)}
\yolozu{} supports the ingestion of a YAML or JSON configuration file as its primary positional argument.
Internally, \cmd{yolozu train} delegates the \cmd{--config <file>} argument to \path{rtdetr_pose.train_minimal}, whereas \cmd{yolozu test} maps YAML keys directly to \path{yolozu.scenarios_cli} arguments.

<begin{longtable}{@{}p{0.18\textwidth}p{0.30\textwidth}p{0.46\textwidth}@{=}
\toprule
\textbf{Use Case} & \textbf{YAML File} & \textbf{Command Pattern} \\
\midrule
Train (Default) & \path{configs/examples/train_setting.yaml} & \cmd{yolozu train <train_setting.yaml>} \\
Train (Contract) & \path{configs/examples/train_contract.yaml} & \cmd{yolozu train <train_contract.yaml> --run-id <id>} \\
Resume & Same as Train (Contract) & \cmd{yolozu train <train_contract.yaml> --run-id <id> --resume} \\
Test (Scenario) & \path{configs/examples/test_setting.yaml} & \cmd{yolozu test <test_setting.yaml> [test_args...]} \\
\bottomrule
\end{longtable}

Concrete execution examples:
\begin{lstlisting][language=bash]
yolozu train configs/examples/train_setting.yaml
yolozu train configs/examples/train_contract.yaml --run-id exp01
yolozu train configs/examples/train_contract.yaml --run-id exp01 --resume
yolozu test configs/examples/test_setting.yaml --adapter precomputed --predictions reports/predictions.json
\end{lstlisting}

Backbone substitutions are governed exclusively by the model configuration JSON, rather than by modifying adapter commands directly:
\begin{lstlisting}[language=bash]
yolozu train configs/examples/train_setting.yaml \
  --model-config rtdetr_pose/configs/base.json
\end{lstlisting}

For comprehensive details regarding the backbone contract and supported architectures, refer to \path{docs/backbones.md}.

\subsection{Test YAML and the Override Rule}
The default test YAML configuration is intentionally minimalist (typically specifying only the adapter, dataset, split, and max\_images). Runtime-specific parameters are conventionally supplied as CLI overrides:
\begin{lstlisting][language=bash]
yolozu test configs/examples/test_setting.yaml \
  --adapter rtdetr_pose \
  --config rtetr_pose/configs/base.json \
  --checkpoint /path/to/checkpoint.pt \
  --dataset data/smoke \
  --max-images 50
\end{lstlisting}

\noindent This paradigm directly mirrors the documented real-model workflow detailed in \path{docs/real_model_interface.md}.

\section{Repository Power Tools}
A multitude of task-specific scripts are housed within the \path{tools/} directory. Notable examples include:
\begin{itemize}
  \item \path{tools/validate_predictions.py}
  \item \path{tools/eval_instance_segmentation.py}
  \item \path{tools/eval_segmentation.py}
  \item \path{tools/export_predictions.py} (Facilitates adapter-based export)
\end{itemize}

For an exhaustive inventory, consult \path{tools/manifest.json} (for machine-readable metadata) and the documentation index (for human-readable guidance).
